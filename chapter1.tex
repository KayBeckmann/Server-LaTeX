\chapter{Analyse und Vorbereitung}

\section{Einführung}
% Hier folgt der Einführungstext.

\section{Hardware}
\subsection{Mainboard}
% Beschreibung des Mainboards und der Anforderungen.

\subsection{Festplatten}
% Erläuterungen zur SSD, HDD, RAID-Konzept etc.

\subsection{USV}
% Hinweise zur USV und deren Einsatz im System.

\section{Betriebssystem}
\subsection{Systemanalyse}
\subsubsection{ArchLinux}
% Beschreibung und Vor- und Nachteile von ArchLinux.

\subsubsection{Debian}
% Text in Bearbeitung – Vorteile/Nachteile.

\subsubsection{Ubuntu}
% Beschreibung und Erfahrungen mit Ubuntu.

\subsubsection{Fedora}
% Kurze Erläuterung zu Fedora.

\subsubsection{OpenSuse}
% Gründe, weshalb OpenSuse hier weniger in Betracht gezogen wird.

\subsubsection{Wahl des OS}
% Zusammenfassung der Entscheidungskriterien.

\subsection{RAID}
% Erläuterungen zum Einsatz eines RAID-Systems.

\subsection{LVM}
% Beschreibung der LVM-Konfiguration.

\subsection{IPv4 vs. IPv6}
% Diskussion der beiden Protokollvarianten.

\section{LAN}
\subsection{Serverprinzip}
% Darstellung des Netzkonzepts und des Serverprinzips.

\section{Virtualisierung}
\subsection{KVM}
% Beschreibung der Virtualisierung mit KVM.

\subsection{Docker}
% Erläuterungen zum Konzept von Docker.

\subsection{Wahl der Virtualisierung}
% Entscheidungsfindung zwischen den Virtualisierungslösungen.

\section{Dienste}
\subsection{Secure Shell}
% Einrichtung und Besonderheiten des SSH-Dienstes.

\subsection{Virtual Private Network}
\subsubsection{OpenVPN}
% Beschreibung und Vorteile von OpenVPN.

\subsubsection{Hamachi}
% Erfahrungen und Besonderheiten bei Hamachi.

\subsubsection{RetroShare}
% Erläuterung zu RetroShare und dessen Einsatzmöglichkeiten.

\subsubsection{NeoRouter}
% Kurze Vorstellung von NeoRouter.

\subsubsection{VPNC}
% Beschreibung des VPNC-Clients.

\subsubsection{Tinc}
% Erläuterungen zur dezentralen VPN-Lösung Tinc.

\subsubsection{VNC}
% Überblick zu VNC und alternativen Fernwartungslösungen.

\subsubsection{Wahl des VPN}
% Zusammenfassung der Überlegungen zur VPN-Auswahl.

\subsection{OwnCloud}
% Beschreibung der OwnCloud-Lösung.

\subsection{Rapla}
% Erläuterung zum Termin- und Ressourcenmanagement mit Rapla.

\subsection{LAMP}
\subsubsection{Document Management System}
% Umsetzung des DMS mit OwnDMS.

\subsubsection{Videoüberwachung}
% Text in Bearbeitung.

\subsubsection{Redmine}
% Einführung in Redmine und dessen Einsatz.

\subsection{BackUp für Clients}
% Konzepte zur Datensicherung der Clients.

\subsection{BackUp wichtiger Systemkomponenten des Servers}
% Strategien zur Sicherung kritischer Systemdaten.

\subsection{Virtueller Server als Spielwiese}
% Nutzung des Servers für Test- und Entwicklungszwecke.
