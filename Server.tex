\documentclass[a4paper,12pt]{book}

% Kodierung, Sprache und Schrift
\usepackage[utf8]{inputenc}
\usepackage[T1]{fontenc}
\usepackage[ngerman]{babel}
\usepackage{lmodern}
\usepackage{microtype}

% Grafiken, Tabellen und Hyperlinks
\usepackage{graphicx}
\usepackage{booktabs}
\usepackage{longtable}
\usepackage{hyperref}

% Seitenlayout
\usepackage{geometry}
\geometry{a4paper, left=25mm, right=25mm, top=25mm, bottom=30mm}

% Kopf- und Fußzeilen mit fancyhdr
\usepackage{fancyhdr}
\pagestyle{fancy}
\fancyhf{} % alle Kopf- und Fußzeilenfelder leeren
% Auf geraden Seiten links: Chapter-Titel; auf ungeraden Seiten rechts: Section-Titel
\fancyhead[LE]{\leftmark} 
\fancyhead[RO]{\rightmark}
\fancyfoot[C]{\thepage}
% Für plain Seiten (z. B. Kapitelanfang) Seitenzahl zentriert
\fancypagestyle{plain}{%
  \fancyhf{}%
  \fancyfoot[C]{\thepage}%
}

% Inhaltsverzeichnis auf Deutsch
\renewcommand{\contentsname}{Inhaltsverzeichnis}

\title{Server für klein Betriebe und Heimgebrauch}
\author{Kay Beckmann}
\date{12. März 2025}

\begin{document}

\frontmatter
\maketitle
\tableofcontents

\mainmatter
\chapter*{Vorwort}
\addcontentsline{toc}{chapter}{Vorwort}
Dieses Dokument dient primär zwei Zwecken:
\begin{itemize}
  \item Eine Dokumentation des Servers, um die getroffenen Entscheidungen und Konfigurationen nachvollziehbar zu machen.
  \item Die Möglichkeit, den Umgang mit \LaTeX\ zu erlernen und zu vertiefen.
\end{itemize}
   % Vorwort (unnummeriertes Kapitel)
\chapter{Analyse und Vorbereitung}

\section{Einführung}
% Hier folgt der Einführungstext.

\section{Hardware}
\subsection{Mainboard}
% Beschreibung des Mainboards und der Anforderungen.

\subsection{Festplatten}
% Erläuterungen zur SSD, HDD, RAID-Konzept etc.

\subsection{USV}
% Hinweise zur USV und deren Einsatz im System.

\section{Betriebssystem}
\subsection{Systemanalyse}
\subsubsection{ArchLinux}
% Beschreibung und Vor- und Nachteile von ArchLinux.

\subsubsection{Debian}
% Text in Bearbeitung – Vorteile/Nachteile.

\subsubsection{Ubuntu}
% Beschreibung und Erfahrungen mit Ubuntu.

\subsubsection{Fedora}
% Kurze Erläuterung zu Fedora.

\subsubsection{OpenSuse}
% Gründe, weshalb OpenSuse hier weniger in Betracht gezogen wird.

\subsubsection{Wahl des OS}
% Zusammenfassung der Entscheidungskriterien.

\subsection{RAID}
% Erläuterungen zum Einsatz eines RAID-Systems.

\subsection{LVM}
% Beschreibung der LVM-Konfiguration.

\subsection{IPv4 vs. IPv6}
% Diskussion der beiden Protokollvarianten.

\section{LAN}
\subsection{Serverprinzip}
% Darstellung des Netzkonzepts und des Serverprinzips.

\section{Virtualisierung}
\subsection{KVM}
% Beschreibung der Virtualisierung mit KVM.

\subsection{Docker}
% Erläuterungen zum Konzept von Docker.

\subsection{Wahl der Virtualisierung}
% Entscheidungsfindung zwischen den Virtualisierungslösungen.

\section{Dienste}
\subsection{Secure Shell}
% Einrichtung und Besonderheiten des SSH-Dienstes.

\subsection{Virtual Private Network}
\subsubsection{OpenVPN}
% Beschreibung und Vorteile von OpenVPN.

\subsubsection{Hamachi}
% Erfahrungen und Besonderheiten bei Hamachi.

\subsubsection{RetroShare}
% Erläuterung zu RetroShare und dessen Einsatzmöglichkeiten.

\subsubsection{NeoRouter}
% Kurze Vorstellung von NeoRouter.

\subsubsection{VPNC}
% Beschreibung des VPNC-Clients.

\subsubsection{Tinc}
% Erläuterungen zur dezentralen VPN-Lösung Tinc.

\subsubsection{VNC}
% Überblick zu VNC und alternativen Fernwartungslösungen.

\subsubsection{Wahl des VPN}
% Zusammenfassung der Überlegungen zur VPN-Auswahl.

\subsection{OwnCloud}
% Beschreibung der OwnCloud-Lösung.

\subsection{Rapla}
% Erläuterung zum Termin- und Ressourcenmanagement mit Rapla.

\subsection{LAMP}
\subsubsection{Document Management System}
% Umsetzung des DMS mit OwnDMS.

\subsubsection{Videoüberwachung}
% Text in Bearbeitung.

\subsubsection{Redmine}
% Einführung in Redmine und dessen Einsatz.

\subsection{BackUp für Clients}
% Konzepte zur Datensicherung der Clients.

\subsection{BackUp wichtiger Systemkomponenten des Servers}
% Strategien zur Sicherung kritischer Systemdaten.

\subsection{Virtueller Server als Spielwiese}
% Nutzung des Servers für Test- und Entwicklungszwecke.
  % Kapitel 1: Analyse und Vorbereitung
\chapter{Einrichtung}

\section{Einführung}
% Allgemeine Einführung zur Einrichtung des Servers.

\section{Hardware}
\subsection{Übersicht der Komponenten}
% Detaillierte Auflistung der Hardwarekomponenten (z. B. Tabelle 2.2.1).

\subsection{Serverprinzip}
% Wiederholung des Serverprinzips im Zusammenhang mit der Installation.

\subsection{Zusammenbau}
% Beschreibung des physischen Aufbaus.

\section{Installation des Servers}
\subsection{Installation Host-OS}
% Detaillierte Installationsanleitung (Partitionierung, Dateisysteme, etc.).

\subsection{Basisdienste auf dem Host-Betriebssystem}
% Einrichtung grundlegender Dienste wie SSH, OpenVPN, etc.

\section{Virtuelle Maschinen}
\subsection{LAMP}
% Einrichtung der LAMP-Umgebung.

\subsection{OwnCloud}
% Installation und Konfiguration von OwnCloud.

\subsection{OwnDMS}
% Einrichtung von OwnDMS als Dokumenten-Management-System.
  % Kapitel 2: Einrichtung
\chapter{Planmäßige Materialerhaltung}

\section{Einführung}
% Allgemeine Erläuterung zur Materialerhaltung.

\section{Reinigung der Lüfter}
% Detaillierte Anweisungen zur regelmäßigen Reinigung der Lüfter.
  % Kapitel 3: Planmäßige Materialerhaltung

\end{document}
